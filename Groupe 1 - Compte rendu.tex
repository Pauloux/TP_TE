\documentclass[12pt, a4paper]{article}

\usepackage[T1]{fontenc}
\usepackage[utf8]{inputenc}


\usepackage[french]{babel} %Pour les langues
\usepackage{hyperref} %Pour les liens et les métadonnées
\usepackage[margin=2.5cm]{geometry} %Pour les marges de la page
\usepackage{siunitx} %Pour les unités

\hypersetup{
 colorlinks=true,
 linkcolor=black,
 urlcolor=black,
 pdfauthor={Paul ROUSSEAU, Maxime BELLAUD, Gauthier L'ÉQUILBECQ, Khalil FALAH},
 pdftitle={Groupe 1 - Compte rendu}
} %Pour modifier la couleur des liens et définir les métadonnées

\title{Transferts énergétiques - TP \\
Compte rendu}
\author{Groupe 1 : \\
Paul ROUSSEAU, Maxime BELLAUD, \\
Gauthier L'ÉQUILBECQ, Khalil FALAH}
\date{Mercredi 15 novembre 2023}

\begin{document}

\maketitle

\tableofcontents

\section{Introduction}

L'objectif de ce TP est de réaliser une étude des performances énergétiques du bâtiment G. Pour cela, la classe de TP étant séparé en quatre groupes, chacun réalise l'étude d'une face. Notre groupe, le groupe 1 réalisera l'étude pour la face Nord. Avec les résultats des autres groupes, nous pourront réaliser l'étude pour le bâtiment complet et enfin simuler le flux de chaleur pour différentes situations météorologiques (froid, chaud, neige, vent, etc...).

\section{Étude de la face Nord}

\subsection{Prise des mesures}

\subsection{Calcul du flux thermique}

\section{Étude du bâtiment complet}

\section{Étude pour différentes situations}
%Je n'ai pas la feuille sur moi donc ce sont des exemples
\subsection{Température extérieure de -10°C, temps sec}

\subsection{Température extérieure de -10°C avec une couche de neige sur le toit de 20cm}

\subsection{Température extérieure de +40°C}

\subsection{Température extérieure de +10°C avec 100 km/h de vent}

\section{Conclusion}

\end{document}
